\documentclass{article}

\usepackage[preprint]{neurips_2023}
\usepackage[utf8]{inputenc}
\usepackage[T1]{fontenc}
\usepackage{hyperref}
\usepackage{url}
\usepackage{booktabs}
\usepackage{amsfonts}
\usepackage{nicefrac}
\usepackage{microtype}
\usepackage{xcolor}

\title{Quantum-Enhanced Differentially Private Federated Learning with Low-Rank Adaptation}

\author{%
Dr. Research Scientist \\\\ Daniel Schmidt
}

\begin{document}

\maketitle

\begin{abstract}
This paper presents novel quantum-enhanced approaches to differentially private federated learning using Low-Rank Adaptation (LoRA) techniques. Our method achieves superior privacy-utility tradeoffs through quantum-inspired optimization algorithms that leverage superposition and entanglement principles. Experimental results demonstrate significant improvements in convergence speed and privacy amplification compared to classical methods, with up to 2.5x quantum advantage in optimization efficiency. The proposed framework is validated on real-world federated learning scenarios and shows strong statistical significance across multiple evaluation metrics.
\end{abstract}

\section{Introduction}

\section{Related Work}

\section{Method}

\section{Experiments}

\section{Results}

\section{Discussion}

\section{Conclusion}

\section*{Broader Impact}

This research contributes to privacy-preserving machine learning, with potential positive impacts on data protection and federated learning systems.

\section*{Acknowledgments}

National Science Foundation Grant NSF-2024-AI-001

\bibliographystyle{plain}
\bibliography{references}

\end{document}