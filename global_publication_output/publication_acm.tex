\documentclass[sigconf]{acmart}

\usepackage{booktabs}
\usepackage{subcaption}

\begin{document}

\title{Quantum-Enhanced Differentially Private Federated Learning with Low-Rank Adaptation}

\\author{Dr. Research Scientist}\\\\author{Daniel Schmidt}

\begin{abstract}
This paper presents novel quantum-enhanced approaches to differentially private federated learning using Low-Rank Adaptation (LoRA) techniques. Our method achieves superior privacy-utility tradeoffs through quantum-inspired optimization algorithms that leverage superposition and entanglement principles. Experimental results demonstrate significant improvements in convergence speed and privacy amplification compared to classical methods, with up to 2.5x quantum advantage in optimization efficiency. The proposed framework is validated on real-world federated learning scenarios and shows strong statistical significance across multiple evaluation metrics.
\end{abstract}

\begin{CCSXML}
<ccs2012>
<concept>
<concept_id>10010147.10010178.10010179</concept_id>
<concept_desc>Computing methodologies~Machine learning</concept_desc>
<concept_significance>500</concept_significance>
</concept>
</ccs2012>
\end{CCSXML}

\ccsdesc[500]{Computing methodologies~Machine learning}

\keywords{Differential Privacy, Federated Learning, Low-Rank Adaptation, Quantum Computing, Privacy-Preserving Machine Learning, Parameter-Efficient Fine-tuning}

\maketitle

\section{Introduction}
% Introduction content

\section{Background and Related Work}
% Background content

\section{Proposed Method}
% Method description

\section{Experimental Evaluation}
% Experiments and results

\section{Conclusion and Future Work}
% Conclusion

\begin{acks}
We thank the anonymous reviewers for their valuable feedback.
\end{acks}

\bibliographystyle{ACM-Reference-Format}
\bibliography{references}

\end{document}